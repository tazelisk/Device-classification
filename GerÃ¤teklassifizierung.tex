\documentclass[fontsize=11pt,a4paper] {article}
% Präambel
\usepackage[T1]{fontenc}    % Silbentrennung bei Sonderzeichen
\usepackage{ngerman}
\usepackage[utf8]{inputenc}
\usepackage{ulem}
\usepackage{caption}
\usepackage{graphicx}
\usepackage{mwe}
\usepackage{nccmath}
\usepackage{multicol}
\usepackage{amsmath}
\usepackage{mathtools}
\usepackage{color}\definecolor{citation}{rgb}{0.2,0.3,1}
\usepackage[final]{hyperref}
\usepackage{tabularx}
\usepackage{multirow}
\usepackage{listings}
\usepackage{epstopdf}
\usepackage{varwidth}
\usepackage{tikz}
\usepackage{pdfpages}
\usepackage[framed]{matlab-prettifier}
\usepackage{float}
\usepackage{xcolor}
\usepackage{color}															 % 
\usepackage{colortbl}
\usepackage{flafter}
\usepackage[backend=biber,
style=alphabetic,
]{biblatex}
\setlength\parindent{0pt}
% LaTeX
% Begin
\setlength{\hoffset}{-0.4cm}
\setlength{\voffset}{-0.54cm}
\setlength{\topmargin}{0cm}
\setlength{\headheight}{0cm}
\setlength{\headsep}{0cm}
\setlength{\textheight}{25cm}
\setlength{\textwidth}{17cm}
\setlength{\oddsidemargin}{0.2cm}
\setlength{\evensidemargin}{\paperwidth-2in-2\hoffset-\textwidth-\oddsidemargin}
\setlength{\marginparsep}{0cm}
\setlength{\marginparwidth}{0cm}
\setlength{\marginparpush}{0cm}
\setlength{\footskip}{1.3cm}
% End

% Commands
\renewcommand{\figurename}{Abb.}
\addbibresource{literatur.bib}

%
\begin{document}
	
	
	\begin{titlepage}
		\centering
		
		\vspace{2cm}
		
		{\huge \bfseries Medizinproduktentwicklung Grundlagen\par}
		
		\vspace{1cm}
		
		{\LARGE\scshape University of Applied Sciences Lucerne\par}
		
		\vspace{1cm}
		
		
		{\large\scshape Department of Medical Engineering}
		
		\vspace{2cm}
		
		{\Large\scshape Testat \par}
		
		{\Large \bfseries Geräteklassifizierung \par}
		
		\vspace{2.5cm}
		
		Benjamin Alain \textsc{Zeliska}\\[1.5cm]

		
		
		
		\emph{Lecturer:} \\ [0.3cm]
		{Roger  \textsc{Abächerli} }\\[4cm]
		
		\href{https://github.com/}{Version 1.0} 
		\vfill
		
		% Bottom of the page
		{\large \today\par}
	\end{titlepage}
	%
	\tableofcontents
	%
	\newpage
	%
	
	\section{Klassifikationsgruppen nach MDR 2017/745}
	
	 \begin{table}[H]
	 	\caption[Grundlagen nach MDR 2017/745]{Grundlagen der Klassifikation} 
	 	\begin{tabular}{||m{4cm}|m{4cm}|m{4cm}|m{4cm}|} \hline
	 		\textbf{Nicht-Invasiv}	& \textbf{Invasiv} & \textbf{Aktiv} & \textbf{Sonderfälle} 
	 		\\\hline
	 		Regel \textbf{1 - 3} & Regel \textbf{5 - 8} & Regel \textbf{9 - 13} & Regel \textbf{14 - 22} \\\hline
	 	
	 	\end{tabular}
	 \end{table}
 %%%%
  \section{Geräteklassifizierung nach MDR 2017/745}
  \subsection{Bluttransfusionsbeutel}
\begin{itemize}
	\item \textbf{Medizinprodukt?}
	\subitem JA, nach Kapitel III, \textit{Anhang VIII} der MDR
	\item \textbf{Nicht-Invasiv} 
	\subitem Regelsatz 1 - 3
	\item \textbf{Angewandte Regel:} 2
	\subitem Blutbeutel gehören zur Klasse IIb.
\item \textbf{Anmerkung}
\subitem Der Blutbeutel ist ein \textit{Sonderfall} und wird daher explizit in der MDR erwähnt.
\end{itemize}
%%%%
\subsection{Magnetresonanztomograph (MRI)}
\begin{itemize}
\item \textbf{Medizinprodukt?}
\subitem JA, nach Kapitel I, Unterkapitel 2.5, \textit{Anhang VIII} der MDR
\item \textbf{Aktiv} 
\subitem Regelsatz 9 - 13
\item \textbf{Angewandte Regel:} 10
\subitem Aktive Produkte zu Diagnose- und Überwachungszwecken gehören zur Klasse IIa.
\item \textbf{Anmerkung}
\subitem Die Funktionsweise eines MRI erfolgt ohne ionisierende Strahlung und ohne sichtbaren Spektralbereich.
\end{itemize}
%%%%
\subsection{Continuous Glucose Monitoring (CGM)}
\begin{itemize}
	\item \textbf{Medizinprodukt?}
	\subitem JA, nach Kapitel I, Unterkapitel 2.5, \textit{Anhang VIII} der MDR
	\item \textbf{Aktiv} 
	\subitem Regelsatz 9 - 13
	\item \textbf{Angewandte Regel:} 10
	\subitem Art der Änderung dieser Parameter könnte zu einer unmittelbaren Gefahr für den Patienten führen,\dots, in diesen Fällen werden sie der Klasse IIb zugeordnet.
	\item \textbf{Anmerkung}
	\subitem Ein CGM misst den Glukosegehalt des Gewebes.
\end{itemize}










	\newpage
	
	%%%%%
	%%%%ABBILDUNGSVERZEICHNIS%%%%%
%	\listoffigures% Abbildungsverzeichnis
	%%%%%%TABELLENVERZEICHNIS%%%%%
	\listoftables% Tabellenverzeichnis
	
	
	%%%%
	\printbibliography
	
	%%%
	% \cite{beispiel}
	
	%%%%%%%% LITERATURVERZEICHNIS %%%%%%%%%%%
	%%


	
	%\flushleft
	%\begin{figure}[ht]
	%	\centering
	%	\includegraphics[scale=0.40]{test.jpg} 
	%	\caption[Beschreibung im Inhaltsverzeichnis]{Beschreibung unter dem Bild.}
	%	\label{fig:nettop}
	%\end{figure}
	
\end{document}